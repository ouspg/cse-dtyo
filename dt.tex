%
% Template for Department of Electrical and Information Engineering Diploma Thesis v1.1.2013
% Authors: Mika Korhonen (original author), Pekka Pietikäinen, Christian Wieser, Teemu Tokola and Juha Kylmänen.
% If you make any improvements to this template, please contact ouspg@ee.oulu.fi
%

\documentclass[a4paper, 12pt,titlepage]{dithesis}
\usepackage[english,finnish]{babel}
\usepackage[utf8]{inputenc}
\usepackage[T1]{fontenc}  
\usepackage{times}
\usepackage{tabularx}
\usepackage{graphicx}
\usepackage{float}
\usepackage{enumerate}
\usepackage{placeins}
\usepackage{fancybox}
\usepackage{verbatim}
\usepackage{longtable}
\usepackage{di}
\usepackage[hyphens]{url}
\usepackage{boxedminipage}
\usepackage{subfigure}
\usepackage{multirow}
\usepackage{amsfonts}
\usepackage{xcolor}
\tolerance=500

%\usepackage[a4paper,margin=2.5cm,dvips]{geometry}
%\geometry{papersize={210mm,297mm}}
%dvipdf -sPAPERSIZE=a4

% The following code removes %-signs with URL:s longer than 72 chars
\begingroup
\makeatletter
\g@addto@macro{\UrlSpecials}{%
  \endlinechar=13 \catcode\endlinechar=12
  \do\%{\Url@percent}\do\^^M{\break}}
 \catcode13=12 %
 \gdef\Url@percent{\@ifnextchar^^M{\@gobble}{\mathbin{\mathchar`\%}}}%
\endgroup %

%\selectlanguage{finnish}

\otsikko{Suomenkielinen nimi}
\title{English title}

\etunimi{Teemu}
\sukunimi{Teekkari}
\valvoja{prof. Juha Röning}
\koulutusohjelma{information} % {information | electrical}
\vuosi{20xx}
\tyo{Master} % {Bachelor | Master}
\kieli{english} % {finnish | english}

\begin{document}

%% Computer science and engineering
%\iffalse     % comment multiple lines with \iffalse, \fi -pair
    \begin{titlepage}
        \centering{\includegraphics*[width=0.3\textwidth]{uni_logo}\\}
        {\sffamily\fontsize{9}{1pt}\selectfont FACULTY OF INFORMATION TECHNOLOGY AND ELECTRICAL ENGINEERING\\}
        %{\sffamily\fontsize{9}{1pt}\selectfont TIETO- JA SÄHKÖTEKNIIKAN TIEDEKUNTA\\}
        \vspace{65 mm}
        {\textbf{\fontsize{16}{19pt}\selectfont \getfirstname\ \getlastname }\\}
        \vspace{15 mm}
        {\textbf{\fontsize{18}{22pt}\selectfont FIRST FRONT PAGE TITLE LINE\\ADDITIONAL IF NEEDED\\}}
        \vspace{60 mm}
        {\fontsize{14}{17}\selectfont Master's Thesis \\Degree Programme in Computer Science and Engineering \\ Month 20xx\\}
        %{\fontsize{14}{17}\selectfont Diplomityö \\Tietotekniikan tutkinto-ohjelma \\ Month 20xx\\}
    \end{titlepage}
%\fi

%% Electrical engineering
\iffalse     % comment multiple lines with \iffalse, \fi -pair
    \begin{titlepage}
        \centering{\includegraphics*[width=0.25\textwidth]{uni_logo}\\}
        {\sffamily\fontsize{8}{10}\selectfont DEGREE PROGRAMME IN ELECTRICAL ENGINEERING\\}
        %{\sffamily\fontsize{8}{10}\selectfont SÄHKÖTEKNIIKAN KOULUTUSOHJELMA\\}
        \vspace{50 mm}
        {\textbf{\fontsize{20}{24}\selectfont MASTER'S THESIS }\\}
        %{\textbf{\fontsize{20}{24}\selectfont DIPLOMITYÖ }\\}
        \vspace{18 pt}
        {\textbf{\fontsize{18}{22}\selectfont TITLE IS WRITTEN HERE \\ ADDITIONAL LINE IF NEEDED \\}}
        \vspace{60 mm}
        {\fontsize{12}{15}\selectfont
            \begin{table}[ht]
                \centering
                \begin{tabular}{p{4.5cm}l}
                    Author & \getfirstname\ \getlastname \\ \\
                    %Tekijä & Etunimi sukunimi \\ \\
                    Supervisor & Firstname Surname \\ \\
                    %Valvoja & Etunimi sukunimi \\ \\
                    Second Examiner & Firstname Surname \\ \\
                    %Toinen tarkastaja & Etunimi sukunimi \\ \\
                    (Technical advisor & Firstname Surname) \\
                    %(Työn tekninen ohjaaja & Etunimi sukunimi) \\
            \end{tabular}
        \end{table}}
        {\fontsize{12}{15}\selectfont Month 20xx}
    \end{titlepage}
\fi


\selectlanguage{english}

\begin{abstract}
This is a sample abstract.

\keywords sample, keywords

\end{abstract}

\selectlanguage{finnish}
\begin{tiivistelma}
Esimerkkitiivistelmä

\avainsanat esimerkki, sanoja
\end{tiivistelma}

\selectlanguage{english}
%\selectlanguage{finnish}

\sisluettelo
%\tableofcontents

\otsake{FOREWORD}
This \LaTeX -template has been used by various people at department
since the late 1990's, and has slowly improved over time.  It is still
somewhat rough at the edges, but hopefully will be helpful in reducing
some of the pain involved in writing a diploma thesis.

Contributors to the template include Mika Korhonen (original author),
Pekka Pietikäinen, Christian Wieser and Teemu Tokola.  If you make any
improvements to this template, please contact ouspg@ee.oulu.fi, and we
will try to include them in further revisions.

The template was updated during the summer of 2013 by Juha Kylmänen.
%\allekirjoitus{Oulu, Finland \today}

\otsake{ABBREVIATIONS}

\setlongtables
\begin{longtable}[l]{p{3cm}p{0.7\textwidth}}

% Add your abbreviations to abbreviations.tex
\input{abbreviations}

\end{longtable}
\setcounter{table}{0}

%Johdanto
\chapter{Introduction}
\sivunumerot
\thispagestyle{empty}
\input{introduction} % ./introduction.tex

\chapter{Implementing something}
\input{implementation}  % ./implementation.tex

\chapter{Testing something}
\input{testing}  % ./testing.tex

\chapter{Discussion}
\input{discussion}  % ./discussion.tex

\chapter{Conclusion}
\input{conclusion}  % ./conclusion.tex

\bibliographystyle{di}
\bibliography{di}
\end{document}
